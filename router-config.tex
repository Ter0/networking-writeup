\chapter{Router Configuration}
\section{Static Routing} Static routing is a
method of explicitly stating where a router should send its traffic for a given
network. Due to it's Administrative Distance of 1 a static route is chosen over
any other routing protocol, only being overruled if the router can see that the
destination address is on a local interface (directly connected). Once the BT
network began to adopt IS-IS as its interior gateway protocol (IGP) static
routes were removed in favour of dynamic routing, they were also removed from
the configurations to ensure that static routes were not erroneously chosen
over a IS-IS sourced counterpart.

\begin{lstlisting}[caption={Alpha Static Route}, label={listing:alpha-static}]
ip route 10.1.1.1 255.255.255.255 100.100.105.1
\end{lstlisting}

In the final topology of BT a single static route has remained in use, the
configuration of which is shown in Listing~\ref{listing:alpha-static}. This
route is used as a component of the eBGP peer link to Autonomous System No. 42,
this peering is sourced from a Loopback interface on the remote side, because
of this the local side requires a static route which provides initial
connectivity to this Loopback. Without the use of these static routes the BGP
sources on both sides would not be able to communicate and thus will never peer
with one- another.

\section{Interior Gateway Protocols}
Within the three-router layout it can be seen that all networks within the
topology has at least two paths to connect to any other network, due to this
fact it was decided that the use of static routing would not provide the fault-
tolerance that would be expected in an ISP network. To provide this additional
level of redundancy we opted to use an IGP to provide automatic routing within
the BT infrastructure. This routing is provided using Intermediate System -
Intermediate System (IS-IS), a link state protocol which computes the shortest
path between devices in the network using Dijkstra's Shortest Path Algorithm. We
have opted to use dynamic routings in BT's internal infrastructure due to its
reliability and fast convergence time when any issues arise within the network.

IS-IS is implemented using the Loopback interfaces defined on each router in the
network, from the IP Address of these Loopbacks we have defined a NET address
which represents the device in the IS-IS routing process. The NET is an NSAP
format which defines information about the device such as the Area ID and a
unique ID for the device in that Area. An example of the NET address for Alpha
is shown in Listing~\ref{listing:alpha-isis}, from this we can see that Alpha is
a member of Area 1, having an IP address of \texttt{12.1.1.1}.

\begin{lstlisting}[caption={Alpha IS-IS Configuration}, label={listing:alpha-isis}]
hostname Alpha 
!  
router isis net 49.0001.0120.0100.1001.00  
is-type level-1  
passive-interface FastEthernet0/0/1  
\end{lstlisting}

In addition to the NET configuration, for IS-IS to include networks in its
routing table the network needs to be added on an interface-by-interface basis,
flagging the interface to be included in the routing table. This has been
implemented on all interfaces inside of BT on both IPv4 and IPv6. Once both the
routing process and the interfaces for routing has been defined on all devices
the IS-IS process runs the shortest path algorithm and selects the best routes
within the network. When connecting BT to other Autonomous Systems for BGP it
soon became apparent that other groups had opted for a similar NET address
schema as ourselves, this caused additional problems as routes from another
ISP's IS-IS process were being passed into the BT infrastructure. In order to
stop these routes from coming into the internal routing process passive
interfaces were configured on all external links, allowing the interface to
remain in the IS-IS routing table but preventing route updates from being sent
or received on that link.

\section{Exterior Gateway Protocols}
\section{Security}

\section{Traffic Management}
In order to simulate a real world situation, traffic entering and leaving our
network needs to be managed. In order to demonstrate this, AS 42 has been setup
to be our provider, whilist DT and Sky are setup as peers to our ISP. As AS 42
is setup to be our provider, our prefix (12.0.0.0/8) and our peers' prefixes,
Sky and DT (8.0.0.0/8 and 34.0.0.0/8 respectively), are advertised to it. This
means AS 42 will be able to send traffic to our network and to our peers. As
Sky and DT are setup to be peers to our ISP, they will be able to access our
network but will not be able to access each other through us. In a real world
situation, our peers should not be able to access our provider through us but
we have been advised, and also for the sake of providing our peers redunduncy
if their link to AS 42 is disrupted, to allow them to access our provider
through our network. In order to for this to work, only the 10.0.0.0/8 and
12.0.0.0/8 prefixes are advertised to our peers. This will allow the peers to
access our network and AS 42, but not each other.