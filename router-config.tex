\chapter{Router Configuration} \section{Static Routing} Static routing is a
method of explicitly stating where a router should send its traffic for a given
network. Due to it's Administrative Distance of 1 a static route is chosen over
any other routing protocol, only being overruled if the router can see that the
destination address is on a local interface (directly connected). Once the BT
network began to adopt IS-IS as its interior gateway protocol (IGP) static
routes were removed in favour of dynamic routing, they were also removed from
the configurations to ensure that static routes were not erroneously chosen over
a IS-IS sourced counterpart.

In the final topology of BT, a single static route has remained in use. This
route is used as a component of the eBGP peer link to Autonomous System No. 42,
this peering is sourced from a Loopback interface on the remote side, because of
this the local side requires a static route which provides initial connectivity
to this Loopback. Without the use of these static routes the BGP sources on both
sides would not be able to communicate and thus will never peer with one-
another.

\section{Interior Gateway Protocols} section{Exterior Gateway Protocols}
\section{Security}
