\chapter{Router Configuration} \section{Static Routing} Static routing is a
method of explicitly stating where a router should send its traffic for a given
network. Due to it's Administrative Distance of 1 a static route is chosen over
any other routing protocol, only being overruled if the router can see that the
destination address is on a local interface (directly connected). Once the BT
network began to adopt IS-IS as its interior gateway protocol (IGP) static
routes were removed in favour of dynamic routing, they were also removed from
the configurations to ensure that static routes were not erroneously chosen over
a IS-IS sourced counterpart.

In the final topology of BT, a single static route has remained in use. This
route is used as a component of the eBGP peer link to Autonomous System No. 42,
this peering is sourced from a Loopback interface on the remote side, because of
this the local side requires a static route which provides initial connectivity
to this Loopback. Without the use of these static routes the BGP sources on both
sides would not be able to communicate and thus will never peer with one-
another.

\section{Interior Gateway Protocols} section{Exterior Gateway Protocols}
\section{Security}

\section{Traffic Management}
In order to simulate a real world situation, traffic entering and leaving our
network needs to be managed. In order to demonstrate this, AS 42 has been setup
to be our provider, whilist DT and Sky are setup as peers to our ISP. As AS 42
is setup to be our provider, our prefix (12.0.0.0/8) and our peers' prefixes,
Sky and DT (8.0.0.0/8 and 34.0.0.0/8 respectively), are advertised to it. This
means AS 42 will be able to send traffic to our network and to our peers. As Sky
and DT are setup to be peers to our ISP, they will be able to access our network
but will not be able to access each other through us. In a real world situation,
our peers should not be able to access our provider through us but we have been
advised, and also for the sake of providing our peers redunduncy if their link
to AS 42 is disrupted, it was decided that they will be able to access our
provider through our network. In order to for this to work, only the 10.0.0.0/8
and 12.0.0.0/8 prefixes are advertised to our peers. This will allow the peers
to access our network and AS 42, but not each other.
