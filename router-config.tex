\chapter{Router Configuration}
\section{Static Routing} Static routing is a
method of explicitly stating where a router should send its traffic for a given
network. Due to it's Administrative Distance of 1, a static route is chosen
over any other routing protocol, only being overruled if the router can see
that the destination address is on a local interface (directly connected). Once
the BT network began to adopt IS-IS as its interior gateway protocol (IGP)
static routes were removed in favour of dynamic routing. They were also removed
from the configurations to ensure that static routes were not erroneously
chosen over a IS-IS sourced counterpart.

\begin{lstlisting}[caption={Alpha Static Route}, label={listing:alpha-static}]
ip route 10.1.1.1 255.255.255.255 100.100.105.1
\end{lstlisting}

In the final topology of BT a single static route remained in use, the
configuration of which is shown in Listing~\ref{listing:alpha-static}. This
route was used as a component of the eBGP peer link to Autonomous System No.
42. The peering was sourced from a Loopback interface on the remote side and
    because of this the local side required a static route which provided
    initial connectivity to the Loopback. Without the use of these static
    routes the BGP sources on both sides would not have been able to
    communicate and thus would never have peered with one-another.

\section{Interior Gateway Protocols}
Within the three-router layout it can be seen that all networks within the
topology have at least two paths to connect to any other network. Because of
this it was decided that the use of static routing would not provide the
fault-tolerance that would be expected in an ISP network. To provide this additional
level of redundancy we opted to use an IGP to provide automatic routing within
the BT infrastructure. This was done using Intermediate System - Intermediate
System (IS-IS), a link state protocol which computes the shortest path between
devices in the network using Dijkstra's Shortest Path Algorithm. We opted to
use dynamic routings in BT's internal infrastructure due to their reliability
and fast convergence time when any issues arise within the network.

IS-IS was implemented using the Loopback interfaces defined on each router in
the network. From the IP Address of these Loopbacks we have defined a NET
address which represents the device in the IS-IS routing process. The NET is in
NSAP format which defines information about the device such as its Area ID and
a unique ID to identify it within that area. An example NET address for
Alpha is shown in Listing~\ref{listing:alpha-isis}, from this we can see that
Alpha was a member of Area 1 and had an IP address of \texttt{12.1.1.1}.

\begin{lstlisting}[caption={Alpha IS-IS Configuration}, label={listing:alpha-isis}]
hostname Alpha
!
router isis
 net 49.0001.0120.0100.1001.00
 is-type level-1
 passive-interface FastEthernet0/0/1
\end{lstlisting}

For IS-IS to include networks in its routing table the network also needs to be
added on an interface-by-interface basis, flagging the interface to be included
in the routing table. This was implemented on all interfaces inside of BT for
both IPv4 and IPv6. Once both the routing process and the interfaces for
routing has been defined on all devices the IS-IS process runs the shortest
path algorithm and selects the best routes within the network. When connecting
BT to other Autonomous Systems for BGP it soon became apparent that other
groups had opted for a similar NET address schema as ourselves, this caused
additional problems as routes from another ISP's IS-IS process were being
passed into the BT infrastructure. In order to stop these routes from coming
into the internal routing process passive interfaces were configured on all
external links. This allowed the interface to remain in the IS-IS routing table
and prevented route updates from being sent or received on that link.

\section{Exterior Gateway Protocols}
In order for BT to provide internet access to their customers they must have a
complete internet routing table from which they can identify the best possible
routes for their customer's traffic to reach its destination over. In addition
to having knowledge of an internet routing table, BT must also provide
alternate paths to a destination should a route go offline or their links
become saturated with network traffic. In order to provide both of these
services an EGP must be used, of which the only option currently in use is the Border
Gateway Protocol (BGP). This external routing protocol allows Autonomous
Systems to interact with one-another through the exchange of prefixes (network
advertisements) or through traffic management in the form of BGP policy.

The implementation of BGP can be further broken down into two separate forms,
eBGP and iBGP. The first of these handles the peering and advertisements
between Autonomous Systems, facilitating the distribution of the BGP Prefix
table and the configuration of routing metrics on these prefixes in order to
influence routing decisions. In comparison, the iBGP process handles the
propagation of the prefix table throughout internal routers.

For BT's peering to other ISPs we opted to take the approach of peering with
the source of a Loopback interface rather than the traditional method of using
the source interface of the connection. This decision was taken at an early
stage of the BGP peering process with the intention of introducing redundant
connections to peers or implementing multi-hop peering to other Autonomous
Systems. However these plans were not implemented due to a lack of time and
physical interfaces on some routers. Implementation of these connections would
of required additional configuration on our Peer's networks as they would need
to configure a static route on their device in order to connect to our Loopback
and begin the peering process. A snippet of our neighbour peering configuration
is shown in Listing~\ref{listing:sky-ebgp-conf}. It can be seen that in
addition to updating the source of the BGP process it was also necessary to add
the command \texttt{disable-connected-check} which removes a test for BGP peers
to ensure that they are directly connected. However, from our experience in the
lab we found that this command alone did not allow a peering to form. The
additional command \texttt{ebgp-multihop 2} was also required in order for the
peers to form correctly.

\begin{lstlisting}[caption={BT-Sky eBGP Configuration}, label={listing:sky-ebgp-conf}]
router bgp 2110
!
! Omitted
!
 neighbor 12.11.0.2 remote-as 5607
 neighbor 12.11.0.2 description PEER_TO_GROUP1
 neighbor 12.11.0.2 ebgp-multihop 2
 neighbor 12.11.0.2 disable-connected-check
 neighbor 12.11.0.2 update-source Loopback0
!
\end{lstlisting}

All three routers in the topology were configured to distribute their eBGP
routes to one-another through iBGP. These internal peers were again formed on
Loopback addresses as a countermeasure for link failure between the routers
tearing down peer relationships. The use of a Loopback allows the peering to
remain up after IS-IS converges, in comparison using the interface address
would cause the peering to fail as the IP address would not be reachable after
a link failure. The command \texttt{next-hop-self} was used to tell the local
iBGP process to label any routes with their own address as the next hop.
Without this command the next-hop on any routes would remain as the next-hop
for the sending router and this will cause connectivity issues. A snippet of
the iBGP peering configuration between Bravo and Alpha is shown in
Listing~\ref{listing:bravo-ibgp-conf}

\begin{lstlisting}[caption={Bravo-Alpha iBGP Configuration}, label={listing:bravo-ibgp-conf}]
router bgp 2110
 neighbor 12.1.1.1 remote-as 2110
 neighbor 12.1.1.1 update-source Loopback0
 neighbor 12.1.1.1 next-hop-self
\end{lstlisting}

The contents of the prefix table that was distributed by our AS was dependent
on the configured network statements under the BGP process. This allowed us to
dictate which local BT networks we would like our peers to know about and
therefore allow their traffic to reach. In our configurations we decided that
the prefixes we sent should be as specific as possible and should not advertise
unnecessary sections of our network, such as internal point-to-point links.
From this decision we were able to narrow the advertised networks down to a
list of six unique prefixes which were in two groups, the first being the
Customer Segment networks which housed webservers, DNS and our customers. The
second grouping was our router subnets, which would facilitate eBGP multihop
peers as discussed earlier. The configuration for these network advertisements
is shown in Listing~\ref{listing:bravo-net-conf}.

\begin{lstlisting}[caption={Bravo BGP Network Configuration}, label={listing:bravo-net-conf}]
router bgp 2110
 network 12.0.1.0 mask 255.255.255.0
 network 12.0.2.0 mask 255.255.255.0
 network 12.0.3.0 mask 255.255.255.0
 network 12.1.1.1 mask 255.255.255.255
 network 12.2.2.2 mask 255.255.255.255
 network 12.3.3.3 mask 255.255.255.255
\end{lstlisting}

\section{Security}
The first security measure we used was having a user account for every router
and laptop in our setup. For ease of use all users were called ``group2''. This
added the first layer of security, meaning a password was required to access
the devices.

For routers, a second ``enable'' password was enforced. This was used to
prevent users entering enable mode on the routers and set with the command
\texttt{enable password \#\#\#\#\#}. The command \texttt{service password-encryption}
was then used to encrypt the passwords so they did not appear in
the router configuration files as plaintext. It should be noted that
\texttt{service password-encryption} uses a weak form of
encryption~\cite{ciscocracker}, and so is best used as a deterrent for issues
like shoulder-surfing. The alternate option would have been to use
\texttt{enable secret \#\#\#\#\#} instead which stores the password as an MD5
hash. However seeing as MD5 is fairly weak anyway~\cite{md5} we didn't feel it
would make much difference.

Additionally telnet was also disabled over router VTY lines. This was done
because telnet sends login information in plaintext, and so is inherently
insecure. Instead we used SSH with RSA keys to allow secure passwordless
access. Note that this only made SSH authentication passwordless, both router
passwords remained in place.

\section{Traffic Management}
In order to simulate a real world situation, traffic entering and leaving our
network needed to be managed. To demonstrate this, AS 42 was setup to
be our provider, whilst DT and Sky were setup as peers to our ISP.

As our provider, AS 42 should have been able to see routes to our network and to
our peers' networks. In order to do this, our prefix (\texttt {12.0.0.0/8}) and
our peers' prefixes, Sky and DT (\texttt{8.0.0.0/8} and
\texttt{34.0.0.0/8} respectively), were advertised to it.

In order to perform this filter, an access list was defined on Alpha to match
all routes that did not belong to the prefix of \texttt{10.0.0.0/8}, i.e.\
routes not local to AS 42. A route map was then used to tag these routes with
the well known community tag no-export. This stopped the router from
advertising routes to anything that was not on our provider's network.

As Sky and DT were setup to act as peers to our ISP, they should have been able to
access our network but should not have been able to access each other through us. However
for the sake of providing our peers with redundancy if their link to AS 42 was
disrupted, we allowed them to access our provider through our network. In order
for this to work, only the \texttt{10.0.0.0/8} and \texttt{12.0.0.0/8} prefixes
were advertised to our peers. This allowed the peers to see routes to our
network and AS 42, but not each other.

Similar access lists to Alpha were defined on Bravo and Charlie in order to
only advertise prefixes to routes local to our peers. Bravo was connected to
Sky, so routes not on the \texttt{8.0.0.0/8} prefix were tagged with the
community tag no-export. Charlie was connected to DT, so it tagged routes not
on the \texttt{34.0.0.0/8} prefix with the community tag no-export. As our
peers were not allowed to communicate with each other through us, we also had
to define prefix lists on Bravo and Charlie. As they were both allowed to
access our network and our provider, prefix lists were created to permit the
\texttt{10.0.0.0/8} to \texttt{10.0.0.0/32} prefixes and the
\texttt{12.0.0.0/8} to \texttt{12.0.0.0/32} prefixes. This set the routers to
only advertise our network and our provider's network to our peers.

In summary:

\begin{itemize}
    \item Alpha advertised all prefixes to routes local to our provider
    (\texttt{10.0.0.0/8})

    \item Bravo advertised our prefixes (\texttt{12.0.0.0/8} to \texttt
    {12.0.0.0/32}) and our provider's prefixes (\texttt{10.0.0.0/8} to
    \texttt{10.0.0.0/32}) to routes local to Sky (\texttt{8.0.0.0/8})

    \item Charlie advertised our prefixes (\texttt{12.0.0.0/8} to \texttt
    {12.0.0.0/32}) and our provider's prefixes (\texttt{10.0.0.0/8} to \texttt
    {10.0.0.0/32}) to routes local to DT (\texttt{34.0.0.0/8})
\end{itemize}
