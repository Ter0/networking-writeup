\chapter{Laptop Configuration}
\section{DNS}
    DNS is configured for the entire \texttt{12.0.0.0/24} block and \texttt
    {2020:b012::/32} block using the \texttt{.group2} domain. The two DNS servers:
    \texttt{ns1.group2} and \texttt{ns2.group2} are arranged to provide redundancy.
    \texttt{ns1} is hosted on \texttt{laptop-1} and operates as the master
    nameserver. \texttt{ns2} is configured as a slave to the \texttt{ns1} master.
    Appendix~\ref{appendix:bind9} contains the configuration files used to
    configure the DNS software, Bind9. The three zone files found in Listings~\ref
    {listing:master-dns-db-group2},~\ref{listing:master-dns-db-12} and~\ref
    {listing:master-dns-db-2020-b012} provide forward DNS resolution for
    IPv4 and IPv6, reverse DNS resolution for IPv4 and reverse DNS resolution for
    IPv6 respectively. The six devices on the network each have a single canonical
    name, as well as potentially useful alternative names provided as CNAME records.
    For example, the laptop connected to the router \texttt{alpha.group2} has the
    canonical name of \texttt{laptop-1.group2}, as well as a CNAME record for
    \texttt{tftp.group2}.
\section{Webserver}
    Each of the three laptops has a webserver installed. The software used is
    the Apache HTTP Server, available as the package \texttt{apache2} in the
    Ubuntu repositories. Apache requires no additional configuration once
    installed, and immediately begins serving the contents of the \texttt
    {/var/www/} folder on port 80.
\section{Configuration Backups}
\section{Email}