\chapter{Laptop Configuration}
\section{DNS}
    DNS is configured for the entire \texttt{12.0.0.0/24} block and \texttt
    {2020:b012::/32} block using the \texttt{.group2.lboro} domain. The two DNS servers:
    \texttt{ns1.group2.lboro} and \texttt{ns2.group2.lboro} are arranged to provide redundancy.
    \texttt{ns1} is hosted on \texttt{laptop-1} and operates as the master
    nameserver. \texttt{ns2} is configured as a slave to the \texttt{ns1} master.
    Appendix~\ref{appendix:bind9} contains the configuration files used to
    configure the DNS software, Bind9. The three zone files found in Listings~\ref
    {listing:master-dns-db-group2},~\ref{listing:master-dns-db-12} and~\ref
    {listing:master-dns-db-2020-b012} provide forward DNS resolution for
    IPv4 and IPv6, reverse DNS resolution for IPv4 and reverse DNS resolution for
    IPv6 respectively. The six devices on the network each have a single canonical
    name, as well as potentially useful alternative names provided as CNAME records.
    For example, the laptop connected to the router \texttt{alpha.group2.lboro} has the
    canonical name of \texttt{laptop-1.group2.lboro}, as well as a CNAME record for
    \texttt{tftp.group2.lboro}.
\section{Webserver}
    Each of the three laptops has a webserver installed. The software used is
    the Apache HTTP Server, available as the package \texttt{apache2} in the
    Ubuntu repositories. Apache requires no additional configuration once
    installed, and immediately begins serving the contents of the \texttt
    {/var/www/} folder on port 80.
\section{Configuration Backups}
    Configuration backups are performed using TFTP to the host \texttt
    {laptop-1.group2.lboro}, accessible using the alias \texttt
    {tftp.group2.lboro}. The TFTP installation was made using the \texttt
    {tftpd-hpa} package available in the repositories. Provided the service is
    started, a functioning TFTP server is then available to write to \texttt
    {/var/lib/tftpboot}.
\section{Email}
    An email server is hosted on \texttt{laptop-3.group2.lboro}, commonly
    accessed using the \texttt{mail.group2.lboro} alias. The server is setup
    using postfix for SMTP, and dovecot for LMTP. That is,
    postfix handles incoming and outgoing connections, and message
    storage is handed off to dovecot. For security, postfix only supports
    authentication over TLS. The mail server handles mail for the \texttt
    {group2.lboro} domain, allowing a users email to take the form of
    \texttt{username@group2.lboro}. Dovecot is configured to store users mail in
    the \texttt{/var/mail/vhosts} directory, and allow access to the mailboxes
    using IMAP or POP3. Users and their encrypted passwords are specified in
    the \texttt{/etc/dovecot/dovecot-users} file, such as in Listing~\ref
    {listing:dovecot-user}.

    \begin{lstlisting}[caption={Dovecot user}, label={listing:dovecot-user}]
chris@group2.lboro:{MD5-CRYPT}$1$GlwLXh30$fLwPLK47HFd/dGdEgPJSB7
    \end{lstlisting}

    For the purposes of the demonstration, \texttt{laptop-1} and \texttt
    {laptop-2} each have an installation of the terminal email client \texttt
    {mutt} configured to access the \texttt{mail.group2.lboro} server using
    IMAP and SMTP. This allows for the sending of emails between two accounts
    on the network.

