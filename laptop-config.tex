\chapter{Laptop Configuration}
\section{DNS}
    Bind9 ``is the market share leader for domain name system (DNS) server software,
    with something like 85 per cent of installations worldwide.''.~\cite{Mohan2010}
    With such a high level of popularity, it made sense to use Bind9 when configuring
    our own network. DNS was configured for the entire \texttt{12.0.0.0/24} block and \texttt
    {2020:b012::/32} block using the \texttt{.group2.lboro} domain. The two DNS
    servers: \texttt{ns1.group2.lboro} and \texttt{ns2.group2.lboro} were
    arranged to provide redundancy.
    \texttt{ns1} was hosted on \texttt{laptop-1} and operated as the master
    nameserver. \texttt{ns2} was configured as a slave to the \texttt{ns1}
    master. Appendix~\ref{appendix:bind9} contains the configuration files used
    to configure Bind9. The three zone files found in
    Listings~\ref {listing:master-dns-db-group2},~\ref{listing:master-dns-db-12} and~\ref {listing:master-dns-db-2020-b012} provided forward DNS
    resolution for IPv4 and IPv6, reverse DNS resolution for IPv4 and reverse
    DNS resolution for IPv6 respectively. The six devices on the network each
    had a single canonical name, as well as potentially useful alternative
    names provided as CNAME records. For example, the laptop connected to the
    router \texttt{alpha.group2.lboro} had the canonical name of
    \texttt{laptop-1.group2.lboro}, as well as a CNAME record for
    \texttt{tftp.group2.lboro}.
\section{Webserver}
    Each of the three laptops had a webserver installed. The software used was
    the Apache HTTP Server, available as the package \texttt{apache2} in the
    Ubuntu repositories. Apache requires no additional configuration once
    installed, and immediately begins serving the contents of the \texttt
    {/var/www/} folder on port 80.
\section{Configuration Backups}
    Configuration backups were performed using TFTP to the host \texttt
    {laptop-1.group2.lboro}, accessed using the alias \texttt
    {tftp.group2.lboro}. The TFTP installation was made using the \texttt
    {tftpd-hpa} package available in the repositories. Provided the service was
    started, a functioning TFTP server was then available to write to \texttt
    {/var/lib/tftpboot}. For security reasons, permissions on the TFTP server
    were restricted to only allow files to be written if they had been
    explicitly created by the administrator on the machine, such as by
    \texttt{touch alpha-confg}.
\section{Email}
    An email server was hosted on \texttt{laptop-3.group2.lboro}, commonly
    accessed using the \texttt{mail.group2.lboro} alias. The server was setup
    using postfix for SMTP, and dovecot for LMTP. That is, postfix handled
    incoming and outgoing connections, and message storage was handed off to
    dovecot. For security, postfix only supports authentication over TLS. The
    mail server handled mail for the \texttt{group2.lboro} domain, allowing a
    user's email to take the form of \texttt{username@group2.lboro}. Dovecot was
    configured to store users mail in the \texttt{/var/mail/vhosts} directory,
    and allow access to the mailboxes using IMAP or POP3. Users and their
    encrypted passwords were specified in the \texttt{/etc/dovecot/dovecot-users}
    file, such as in Listing~\ref{listing:dovecot-user}.

    \begin{lstlisting}[caption={Dovecot user}, label={listing:dovecot-user}]
chris@group2.lboro:{MD5-CRYPT}$1$GlwLXh30$fLwPLK47HFd/dGdEgPJSB7
    \end{lstlisting}

    For the purposes of the demonstration, \texttt{laptop-1} and \texttt
    {laptop-2} each had an installation of the terminal email client \texttt
    {mutt} configured to access the \texttt{mail.group2.lboro} server using
    IMAP and SMTP. This allowed for the sending of emails between two accounts
    on the network.

