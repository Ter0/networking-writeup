\chapter{Network Design}

\section{Topology}

The topology used for this lab assignment takes the form of a three router set-
up, simulating the infrastructure a typical Internet Service Provider (ISP)
might have. Using three routers as opposed to a one allows the ISP to make use
of additional technologies such as dynamic internal routing and iBGP in order
to create a fault-tolerant network. Using multiple devices results in increased
uptime as traffic can be routed via an alternate path should any outage occur
on any single device.

\begin{figure}[!ht]
    \caption{High-level Topology}
    \centering
    \includegraphics[width=0.8\textwidth]{images/networkTopology.png}
\end{figure}

\section{Address Allocation}
\subsection{IPv4}
For this lab exercise we represent the ISP BT and are assigned the IP block
\texttt{12.0.0.0/8} for our network. This address space provides us with 16.7
million addresses to allocate, of which we require only a small fraction.
However taking into account the issues that arose from the original allocation
of excessive IPv4 network blocks to ISPs and Educational Institutions, we
decided it was best to be conservative in our allocations. With this in mind we
invented three classifications of network address, each defining the minium
subnet size deemed necessary for its purpose. These are outlined in
Figure~\ref{figure:network-alloc-1}.

\begin{figure}[!ht]
    \caption{Classifications of Network Allocations}
    \label{figure:network-alloc-1}
    \centering
    \begin{tabular}{|c|c|p{5.5cm}|}

        \hline
        \textbf{Address Type} & \textbf{Subnet Mask} & \textbf{Justification} \\

        \hline
        Customer Segment & \texttt{/24} & These are allocated to
        downstream customers, who are given enough address space for 254
        devices. In our network, laptops are used on these segments to test
        connectivity to customers. The size of this subnet also allows for easy
        identification of subnets to their location in the topology.\\

        \hline
        Point-to-point Links & \texttt{/30} & This is used for links that
        connect two routers, because for these links only two IP Addresses are
        required and a \texttt{/30} subnet mask is the smallest mask that will
        provide this. \textbf{Note:} Our transit connect to AS 42 is an
        exception to this and uses a \texttt{/24} subnet mask.\\

        \hline
        Loopback Address & \texttt{/32} & Addresses used for the loopback
        interfaces on the routers, there is only a requirement for a single
        address and a \texttt{/32} mask produces this.\\

        \hline
    \end{tabular}
\end{figure}
In addition to the classification of subnets based on size, we used several IP
schemas to choose the IP values of our networks from our allocated block. This
allowed for quick identification of a network from its IP without referencing
our network diagrams. For example, customer segments connected to the Alpha
router have a 3rd octet value of 1, so it is immediatly apparent that the IP
\texttt{12.0.1.0/24} is connected to Alpha.

These conventions were decided in advance of our network build-out and help us
diagnose issues with IP or interior routing configurations in a more intuitive
way. The exact schemas are outlined in Figure~\ref{figure:network-alloc-2}.

\begin{figure}[!ht]
    \caption{IP Schemas}
    \label{figure:network-alloc-2}
    \centering
    \begin{tabular}{|c|p{3cm}|p{6cm}|}
        \hline
        \textbf{Schema} & \textbf{Classification} & \textbf{Identifying Feature} \\

        \hline
        \texttt{12.0.\#.0/24} & Customer Segment & The hash dictates the ISP
        router that this address space is connected to, 1 through 3 for Alpha,
        Bravo and Charlie. This would scale for up to 253 subnets in the
        Service Provider.\\

        \hline

        \texttt{12.\#.0.0/30}, where $\#> 10$ & Peer Point-to-point Link &
        The hash in these networks dictates the assigned number of the group we
        connect to on these links. This enables us to quickly identify the
        group involved with any connectivity issues in BGP. For example, the
        link to group 3 uses the network \texttt{12.13.0.0/30}.\\

        \hline
        \texttt{12.0.0.\#/30} & Internal Point-to-point Link &
        The hash is a multiple of 4, starting from 0. This choice was
        arbitary.\\

        \hline
        \texttt{12.\#.\#.\#/32} & Loopback Address & The hash is the same value
        in all three octets of the address and dictates the router that this
        loopback is assigned to. For example, \texttt{12.3.3.3/32} is the loopback
        address of Charlie.\\
        \hline
    \end{tabular}
\end{figure}
\clearpage

\subsection{IPv6}